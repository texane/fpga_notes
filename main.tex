\documentclass[12pt]{article}

\usepackage{listings}

\title{VHDL notes}
\author{Fabien Le Mentec}
\date{\today}

\begin{document}
\maketitle

\begin{abstract}
Set of practical VHDL notes and guidelines.
\end{abstract}
\newpage

\setcounter{tocdepth}{2}
\tableofcontents
\newpage


\newpage
\section{Reusability}

\subsection{General}
as much as possible, standalone reusable components


\subsection{Use unconstrained types}
use unconstrainted arrays, propagate through hierarchy


\subsection{Use type attributes}
use type attribute as much as possible, esp. type'length
and type'range


\subsection{Use generics, not package constants}
use generics for specialization. esp, think about template
as non modifiable by the user.


\newpage
\section{Simulation}

\subsection{Per component test bench}
use per component simulation test benches, not project wide.
they are easier for the component user to understand, and
force you to design stand alone components.


\newpage
\section{Synthesis}


\subsection{Inference}
\paragraph{}
Usually, a VHDL developer does not explicitely indicate what
hardware resource to use to implement logic. The synthetiser
deduces that from its source code understanding (ie. signal
netlist and operations). This process is known as inference.
\paragraph{}
Inference is very sensitive to the way code is written. For
instance, the use of an additional signal to reset a shift
register may prevent the synthetiser to infer a hardware
shift register.
\paragraph{}
Thus, VHDL developers try as much as possible to write code
in a standard way, that is known to be well understood by the
synthetiser.


\subsection{Explicit resource instanciation}
\paragraph{}
Non portable but sure to instanciate the right resource.


\subsection{Reset signals}
Avoid reset signals. If not possible, make reset clock synchronous.
The following construct helps:
\lstset{language=VHDL}
\begin{lstlisting}[frame=single]
process
begin
 wait until rising_edge(clk);

 if rst = '1' then
 end if;

end process;
\end{lstlisting}



\subsection{Shift registers}
\paragraph{}
XILINX FPGAs have hardware resources to implement shift registers.



\newpage
\section{Verification}


\subsection{Use assertions}
use assertion to check data type lengths when unconstraints
arrays


\newpage
\section{Documentation}

\subsection{Test benches as documentation}
test benches also serve as a documentation for component users


\newpage
\section{Misc}
\subsection{Processes}


\subsection{Clocking}
clear convention about how data passed to/from a component are
clocked. by default, clocked using the component domain. idem for
latching.


\subsection{Typing}
use right types (unsigned, slv ...), sizing and indexing from the
beginning. it avoids further casting and simplifies the code.


\end{document}
