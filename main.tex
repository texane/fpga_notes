\documentclass[12pt]{article}
\usepackage{listings}


\begin{document}

\newcommand{\note}[1]
{\newpage\section{#1}\label{note:#1}}

\newcommand{\topics}[1]
{\paragraph{}\small{\textit{topics: #1}}}

\newcommand{\related}[1]
{\paragraph{}\small{\textit{related notes: \ref{note:#1}}}}

\newcommand{\todo}[1]
{\paragraph{}\textbf{TODO}: #1}

\lstnewenvironment{vhdl}
{\newline\lstset{language=VHDL, frame=single}}{}

\lstnewenvironment{sh}
{\newline\lstset{frame=single}}{}

\IfFileExists{version.tex}
{\input{version.tex}}{\newcommand{\version}{none}}

\title{VHDL notes}
\author{Fabien Le Mentec, fabien.lementec@gmail.com}
\date{\small{version: \version}}


\maketitle

\begin{abstract}
Set of VHDL related notes. It addresses topics ranging from coding
conventions, verification, synthesis, optimisation, reusability
and documentation. Some notes are vague while others are quite
specific to tools or target platforms. Also, it addresses a wide
audience, so some materials may seem obvious to the reader,
depending on her background.
\end{abstract}

\newpage
\setcounter{tocdepth}{1}
\tableofcontents


%%
\note{Unconstrained types}
\topics{reusability, documentation, verification}

\paragraph{}
When not explicitly specified by the developer, a type length
is deduced during component instantiation. This favors component
reusability by letting the user decide of the type length
according to its particular needs.

\paragraph{}
For instance, \textit{count} range is unconstrained in the
following component declaration:
\begin{vhdl}
component my_component
port
(
 ...
 count: in unsigned;
 ...
);
\end{vhdl}

\paragraph{}
The actual range is deduced during instantiation:
\begin{vhdl}
signal count: unsigned(7 downto 0);

work.my_component
port map
(
 ...
 count => count
 ...
);
\end{vhdl}

\paragraph{}
One important issue with unconstraint types is that a component
user may inadvertently use types that are larger than required,
possibly leading to unecessary large resource instantiation.
Documentation is a good tool to solve this kind of issue. Also,
assertions can be used to check for degenerate cases:
\begin{vhdl}
component my_component
port
(
 ...
 -- WARNING
 -- hardware comparator infered in subsequent logic
 -- use appropriate length
 count: in unsigned;
 ...
);
end component;
\end{vhdl}

\paragraph{}
If you want to make sure the length fits within a given range,
use assertion in the component entity definition:
\begin{vhdl}
entity my_component
port
(
 ...
 count: in unsigned;
 ...
);
end my_component;

architecture my_component_rtl of my_component is
begin
...
assert (count'length <= 16)
report "invalid counter length"
severity failure;
...
end my_component_rtl;
\end{vhdl}


%%
\note{Type attributes}
\paragraph{}
Use type attribute as much as possible, esp. type'length
and type'range


%%
\note{Generics instead of package constants}
\topics{reusability}
\paragraph{}
Often, a component parameter can be set either using
generic or package constants. Using package constants
forces the user to modify your package. On the other
hand, generics let the user specializes the component
without modifying any existing source.


%%
\note{Per component test benches}
\topics{simulation}
\paragraph{}
Per component test benches generally requires less code than
project wide ones. It makes them easier to maintain, and
encourages the developer to write self contained components.
Also, it makes simulation run faster.


%%
\note{Hardware resource inference}
\topics{synthesis}
\paragraph{}
Usually, a VHDL developer does not explicitely indicate what
hardware resource to use to implement logic. The synthetiser
deduces that from its source code understanding (ie. signal
netlist and operations). This process is known as inference.
\paragraph{}
Inference is very sensitive to the way code is written. For
instance, the use of an additional signal to reset a shift
register may prevent the synthetiser to infer a hardware
shift register.
\paragraph{}
Thus, VHDL developers try as much as possible to write code
in a standard way, that is known to be well understood by the
synthetiser.


%%
\note{Explicit resource instantiation}
\topics{synthesis}
\todo{wip}
\paragraph{}
Non portable but sure to instanciate the right resource.


%%
\note{Reset signals}
\related{Writing synchronous processes}
\paragraph{}
Avoid reset signals. If not possible, make reset synchronous.
\todo{explain why}


%%
\note{Shift registers inference}
\topics{synthesis}
\todo{wip}
\paragraph{}
XILINX FPGAs have hardware resources to implement shift registers.


%%
\note{Assertions}
\topics{verification}
\todo{wip}

\paragraph{}
Use assertion to check data type lengths when unconstraints
arrays


%%
\note{Test benches as documentation}
\topics{documentation}

\paragraph{}
A component developer should consider test benches an important
part of the documentation since they are used as reference
materials by the component user. Thus, test benches should be up
to date, clearly written and well documented. If possible, they
should cover different use cases, without flooding the user with
unrequired contents.


%%
\note{Writing synchronous processes}
\topics{synthesis}

\paragraph{}
There is one standard way of writing synchronous process:
\begin{vhdl}
process(clk, rst)
begin
 if rising_edge(clk) then
  if rst = '1' then
  else
  end if;
 end if;
end process;
\end{vhdl}

\paragraph{}
Another way which is synthetizable:
\begin{vhdl}
process
begin
 wait until rising_edge(clk);

 if rst = '1' then
 end if;

end process;
\end{vhdl}

\paragraph{}
Since the \textbf{wait} statement must come first, all the signal
are synchronous, esp. the reset. Also, this convention results in
a somewhat clearer code.


%%
\note{Clocking}

\todo{wip}
\paragraph{}
Clear convention about how data passed to/from a component are
clocked. by default, clocked using the component domain. idem for
latching.


%%
\note{Appropriate typing}
\topics{verification, documentation}

\paragraph{}
Use the most specialized types (unsigned, boolean ...) and sizes
early in the design hierarchy. It avoids further casting and
simplifies the code. It acts as documentation since the reader
deduces information from the type itself. For instance, an
unsigned counter tells it can not be negative. Typing also
improve static time checks.


%%
\note{Component directory structure}
\topics{reusability, documentation}

\paragraph{}
Having a clear, self contained directory structure is helpful
for both the user and the developer. I opted for the following
one, that simple but fits most of the cases:
\begin{sh}
my_component/
 src/
  my_component_pkg.vhd
  my_component_rtl.vhd
  ...
 sim/
  common/
   main_tb.vhd
  isim/
   isim.tcl
   isim.prj
   isim.sh
  modelsim/
   ...
 syn/
  ise/
   xc7k325t.ucf
   xc7vx485t.ucf
  vivado/
   ...
 doc/
  my_component.pdf
\end{sh}

\paragraph{}
Providing synthesis files allows the user to synthetise the
component for a given platform. It should not synthetise a full
working design, only the bare minimum so the user can check its
toolchain (esp. version), and investigate what hardware resources
are infered.


\end{document}
